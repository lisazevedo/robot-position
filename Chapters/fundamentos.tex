\chapter{Fundamentação Teórica}
\label{chap:fundteor}

% \begin{flushright}

%    \begin{list}{}{
%       \setlength{\leftmargin}{4.5cm}
%       \setlength{\rightmargin}{0cm}
%       \setlength{\labelwidth}{0pt}
%       \setlength{\labelsep}{\leftmargin}}
%       \item Quanto maior for a rapidez de transformação de uma
%       sociedade, mais temporárias são as necessidades
%       individuais. Essas flutuaçõess tornam ainda mais acelerado
%       o senso de turbilh da sociedade.

%       \begin{list}{}{
%       \setlength{\leftmargin}{0cm}
%       \setlength{\rightmargin}{0cm}
%       \setlength{\labelwidth}{0pt}
%       \setlength{\labelsep}{\leftmargin}}
%       \item (Alvin Toffler)
%       \end{list}
%    \end{list}
% \end{flushright}

% \begin{flushright}
%   Quanto maior for a rapidez de transformação de uma \\
%   sociedade, mais temporárias são as necessidades \\
%   individuais. Essas flutuações tornam ainda mais \\
%   acelerado o senso de turbilhão da sociedade. \\
%   \ \\
%   (Alvin Toffler)
% \end{flushright}

%--------- NEW SECTION ----------------------
\section{Estudo da Odometria}
Informações relacionadas a localização são essenciais em diversas aplicações relacionadas à robótica móvel nos dias atuais. Determinar a posição em um ambiente, dado um mapa daquele ambiente e dados sensoriais locais, pode ser a definição de localização para um robô móvel. Isto foi um dos grandes problemas passados na área de robótica e ainda nos tempos atuais é um campo de bastante estudo.
Como uma das possíveis soluções para este problema, foi apresentado o cálculo baseado em medidas odométricas. Odometria nada mais é que o uso de dados capturados por sensores de movimento e assim então estimar mudança de posição com o tempo. É altamente usada na robótica por alguns robôs com rodas ou legados para estimar sua posição relativa de um ponto de partida. Entretanto, para um uso efetivo desta tecnologia é necessário uma captura rápida e precisa de dados, calibração de instrumentos e processamento.

Um sensor interessante para captura de dados como velocidade e posição é o sensor de WiFi. Um exemplo prático foram pesquisadores da Universidade da Carolina do Norte desenvolveram um meio para capturar a velocidade e distância em ambientes indoor através de um sensor WiFi. Este sensor funciona como um sensor de velocidade para assim rastrear com precisão o quão longe algo se moveu; exatamente como um sonar mas usando ondas de rádio ao invés de ondas de som.


\begin{itemize}

    \item \textbf{Trilateração}: métodos que utilizam as propriedades geométricas do triângulo para encontrar a posição do alvo. Diferente da triangulação, este processo determina o posicionamento a partir de 3 pontos de referência diferentes, assim como acontece em sistemas de GPS(Global Positioning System). Podem ser divididos em métodos por lateração e angulação.;
    
    \begin{itemize}
        \item \textbf{Lateração} estima a posição do objeto através da leitura de distâncias a partir de múltiplos pontos de referência.
        
        \begin{itemize}
            \item \textbf{-TOA (Time Of Arrival)}: a distância do alvo móvel até a unidade medidora
            é diretamente proporcional ao tempo de propagação; 
            \item \textbf{TDOA (Time Difference Of Arrival)}: este método busca determinar a posição relativa do transmissor através da diferença de tempo entre o envio até as unidades recptoras.
            \item \textbf{Baseadas em RSS (Received Signal Strength)}: método que calcula a distância baseada na atenuação da força do sinal entre o emissor e o receptor. Os métodos baseados em RSS, assim como os dois anteriores necessitam da inexistência de obstáculos físicos entre os participantes da conversa.
            \item \textbf{RTOF (Reflection Time Of Flight)}: este método utiliza do valor de Time of Flight, ou tempo de voo do sinal, para estimar a posição. O tempo de voo do sinal 
        \end{itemize}

    \end{itemize}

\end{itemize}

%--------- NEW SECTION ----------------------
\section{Aplicação na Robótica}
\label{sec:ass1}
\lipsum[1]

%---------------picture------------------------------------
% \begin{figure}
%     \centering
%     \subfigure[Figure A]{\label{fig:a}\includegraphics[width=60mm]{./lq}}
%     \subfigure[Figure B]{\label{fig:b}\includegraphics[width=60mm]{./lq}}
%     \subfigure[Figure C]{\label{fig:c}\includegraphics[width=\textwidth]{./lq}}
%     \caption{Three simple graphs}
%     \label{fig:three graphs}
% \end{figure}
%----------------------------------------------------------

