\chapter{Conclusão}
\label{chap:conc}


Com essa pesquisa conclui-se que a aplicação de sensor Wi-Fi para odometria indoor na robotica movel é viavel quando unido com outros sensores, como é mostrado por \cite{Diego2016} que 
utiliza 2 sensores al\'em do sensor Wi-Fi, a camera e o compasso digital. Na pesquisa de \cite{Diego2016} foram realizadas análises com dados reais e dados simulados com relação a precisão de trajetória dos algoritmos envolvidos na odometria visual. As análises da precisão entre trajetória real executada pelo robô e a obtida
pelo robô mostraram uma margem de erro próximo a 2,0 metros que pode ser considerada
aceitável dependendo do tamanho do robô utilizado.

% \textbf{Conclus\~oes}.


% Chegou a hora de apresentar o apanhado geral sobre o trabalho de
% pesquisa feito, no qual s\~ao sintetizadas uma s\'erie de
% reflex\~oes sobre a metodologia usada, sobre os achados e
% resultados obtidos, sobre a confirma\c{c}\~ao ou recha\c{c}o da
% hip\'otese estabelecida e sobre outros aspectos da pesquisa que
% s\~ao importantes para validar o trabalho. Recomenda-se n\~ao
% citar outros autores, pois a conclus\~ao \'e do pesquisador.
% Por\'em, caso necess\'ario, conv\'em cit\'a-lo(s) nesta parte e
% n\~ao na se\c{c}\~ao seguinte chamada \textbf{Conclus\~oes}.


% \section{Considerações finais}
% \label{sec:consid}

% Brevemente comentada no texto acima, nesta se\c{c}\~ao o
% pesquisador (i.e. autor principal do trabalho cient\'ifico) deve
% apresentar sua opini\~ao com respeito \`a pesquisa e suas
% implica\c{c}\~oes. Descrever os impactos (i.e.
% tecnol\'ogicos,sociais, econ\^omicos, culturais, ambientais,
% políticos, etc.) que a pesquisa causa. N\~ao se recomenda citar
% outros autores.

